Topic-aware influence propagation models share three main hypothesis: 
\begin{itemize}
\item[i] Users have different interests
\item[ii] Items have different characteristics
\item[iii] Similar items are likely to interest same users
\end{itemize}
These assumptions entirely describe a propagation cascade through a social influence which is  strictly dependent on the topics and interests. To account preferences we enrich the notion of node defining it with the following triplet:
\newtheorem{defn}{Definition}
\begin{defn}
	(Node)  $\qquad <i,\: \mathbf{f}_{i},\: \mathbf{t}_{i}>$
\end{defn}
 where $i$ is the identification index while $\mathbf{f}_{i}$ and $\mathbf{t}_{i}$ are respectively the influence vector and interests vector of node $i$ in a $K$ dimensional topics space.

\begin{defn} 
(Influence vector) $ \qquad \mathbf{f}_i=[f_i^{1}, f_i^{2}, .. , f_i^{K}]$
\end{defn}

\begin{defn}
(Interests vector) $ \qquad \mathbf{t}_i=[t_i^{1}, t_i^{2}, .. , t_i^{K}]  $
\end{defn}
Each component represents respectively the degree of influence that the node can exerts on a particular topic and the degree of interest of the node in that topic; for these reasons these entries are non-negative defined.

The graph is defined by the set of nodes $V$ and the set of edges $E \subseteq V\times V$: $G=(V,E)$. The topic-aware perspective implies the generalization of its adjacency matrix entry:
\begin{defn}
(Link weight vector) $\textbf{p}_{uv} = [p_{vu}^{1}, p_{vu}^{2}, .., p_{vu}^K]$  
\end{defn}
where $vu$ is the direct link from $u$ to $v$, $z=1,..,K$ is the topic, and
 $p_{vu}^{z} = t_u^z + f_v^z \in [0,+\infty)$. In this way the directed link from node $v$ to $u$ on a particular topic $z$ has an importance proportional to the interest of node $u$ on that topic, incremented by the amount of influence that node $v$ can exert on $u$.

Moreover the necessity of initial conditions in the diffusion process is solved by introducing a special node in the network, called \textit{god node}; it is connected with all other nodes and denoted with a negative index \textit{-1}.
\begin{defn}
(God node) $v_{-1}\:|\:E^+(v_{-1})\equiv V \wedge E^-(v_{-1})\equiv\emptyset$
\end{defn}
where: $E^+(u):= \{v \in V | v \neq u \wedge e_{uv} \in E \}$ is the set of nodes followed by $u$, and symmetrically: $E^-(u):= \{v \in V | v \neq u \wedge e_{vu} \in E \}$.


The corpus $I$, i.e. the set of items with cardinality $I$ involved in the cascades, is described by probability distributions and scalars. As a main concept LDA assumes that each topic is described by a superposition of words while a document can be statistical interpreted by a composition of topics. This particular model defines these distributions as Dirichlet distributions in order to incorporate in the model the assumption that few topics can well describe a text. The document (or \textit{item}) is entirely defined by its distribution over the topic space and a scalar value that defines its propensity to propagate, which we denote as \textit{virality} parameter.

Following the standard notation of the generative LDA model we have:
\begin{itemize}
\item $\mathbf{\alpha}:=$ prior topic distribution
\item $\mathbf{\beta}:=$ prior word distribution
\item $\mathbf{\gamma}:= $ document's distribution over topics; each entry is the probability of topic $z$ occurring in document $i$: $p(k|i)$
\item $\mathbf{\varphi}:=$ topic's distribution over vocabulary words; each entry is the probability of word $w$ occurring in topic $z$: $p(w|z)$ 
\end{itemize} 

As a final step towards cascades modeling we extended the Topic-aware Linear Threshold Model. The activation process is driven by the exceeding of a deterministic threshold. Firstly, we considered the LDA output as item's topic distribution. Secondly, we defined the social pressure on a particular node proportional to her level of interests and the degree of influence of her neighboors. Finally, we bounded the activation parameter introducing a sigmoid function\footnote{$\Theta(x)=1/(1+e^{-x})$}. The activation parameter for node $u$ at time $t$ over item $i$ is:
\begin{equation}
W_i^t (u)= \Theta \left[ \sum_{z=1}^K \left(  \gamma_i^z \sum_{v \in F_{i}(u,t)} p_{vu}^z \right) \right]
\end{equation}
where:
$\gamma_i^z$ is the item's distribution over the topics;
$p_{v,u}^z$ is the strength of influence exerted by $v$ on $u$ on topic $z$;
$F_{i}(u,t)$ is the set of users active on $i$ at time $t$ and that have a link with $u$. If $W_i^t(u)$ is greater or equal to $\theta_i$ threshold, it becomes active item $i$. A couple of considerations: [a] $\gamma_i^z$ is the z-th coefficient of item-$i$'s linear combination of topics; [b] $p_{vu}^z$ corresponds to the link weight of the directed graph from $u$ to $v$ for topic $z$ (\textit{def 4}), remember weights are $K$-dim vectors (one weight for each topic).
Moreover we introduced a set of activation thresholds that is equal for each node: one item has the same threshold for each individual. This quantity is linked to what we call \textit{virality} parameter; in particular the item's threshold is the inverse of its virality. In this way we are introducing the independence of the diffusion strength of a document, which represents the real structural characteristics of the texts. 

The following box summarizes the presented variables:


\framebox[7.5cm]{
 \begin{minipage}{65mm}
  $\textbf{f}_u$, $\textbf{t}_u$:influence and interests vectors of node u\\
  $\alpha$, $\beta$: prior topic, word distribution\\ 
  $\varphi$: topic's distribution over words\\
  $\gamma$: item's distribution over topics\\
  $p_v$: strength of influence exerted by $v$ on an other node; \\
  $W^t$: node's influence weight at time $t$ for an item\\
  $\theta_i$: item $i$ threshold (inverse of its virality)\\
  $W^t_i(u)$: activation parameter for node $u$ at time $t$ on item $i$
 \end{minipage}}
