\usepackage{graphicx}
\usepackage[show]{chato-notes}
\usepackage{amsmath,amssymb}
\usepackage{bbm}
\usepackage{epsfig}
\usepackage{algorithm}
\usepackage{algpseudocode}
\usepackage{url}
\usepackage{hyperref}
\usepackage{tikz}
\usetikzlibrary{bayesnet}

\hypersetup{
    bookmarks=true,         % show bookmarks bar?
    unicode=false,          % non-Latin characters in Acrobats bookmarks
    pdftoolbar=true,        % show Acrobats toolbar?
    pdfmenubar=true,        % show Acrobats menu?
    pdffitwindow=false,     % window fit to page when opened
    pdfstartview={FitH},    % fits the width of the page to the window
    pdftitle={My title},    % title
    pdfauthor={Author},     % author
    pdfsubject={Subject},   % subject of the document
    pdfcreator={Creator},   % creator of the document
    pdfproducer={Producer}, % producer of the document
    pdfnewwindow=true,      % links in new window
    colorlinks=true,       % false: boxed links; true: colored links
    linkcolor=black,          % color of internal links
    citecolor=black,        % color of links to bibliography
    filecolor=black,      % color of file links
    urlcolor=black           % color of external links
}

\interfootnotelinepenalty=10000

%% to save space around floating bodies ( = algorithms, figures, etc.)
%\renewcommand\topfraction{0.85}
%\renewcommand\bottomfraction{0.85}
%\renewcommand\textfraction{0.1}
%\renewcommand\floatpagefraction{0.1}
%\setlength\floatsep{.05\baselineskip plus 3pt minus 4pt}
%\setlength\textfloatsep{.15\baselineskip plus 3pt minus 2pt}
%\setlength\intextsep{1.25\baselineskip plus 3pt minus 2 pt}

\newcommand{\HRule}{\rule{\linewidth}{0.1mm}}
\newcommand{\missing}{\textbf{\textcolor{red}{[missing]}}}

% Paragraphs
\newcommand{\spara}[1]{\smallskip\noindent{\bf #1}}
\newcommand{\mpara}[1]{\medskip\noindent{\bf #1}}
\newcommand{\para}[1]{\noindent{\bf #1}}

%Mathematical environments
%Mathematical environments
\newtheorem{mydefinition}{Definition}
%\newtheorem{proposition}{Proposition}
\newtheorem{myproperty}{Property}
\newtheorem{mytheorem}{Theorem}
%\refstepcounter{mytheorem}
%\newtheorem{mytheoremnoindent}{\!\!\!Theorem}
\newtheorem{mycorollary}{Corollary}
%\newtheorem{claim}{Claim}
\newtheorem{myexample}{Example}
\newtheorem{mylemma}{Lemma}
\newtheorem{problem}{Problem}
\newtheorem{fact}{Fact}
%\newtheorem{assumption}{Assumption}
\newtheorem{observation}{Observation}

\newcommand{\Abs}[1]{\left|#1\right|}
\newcommand{\Tuple}[1]{\left<#1\right>}
\newcommand{\Set}[1]{\left\{#1\right\}}
\newcommand{\List}[1]{\left[#1\right]}
\newcommand{\Paren}[1]{\left(#1\right)}
\newcommand{\Binom}[2]{\left(#1 \atop #2\right)}
\newcommand{\Floor}[1]{\left\lfloor #1 \right\rfloor}
\newcommand{\Ceil}[1]{\left\lceil #1 \right\rceil}
\newcommand{\IntO}[1]{\left(#1\right)}
\newcommand{\IntC}[1]{\left[#1\right]}
\newcommand{\IntLO}[1]{\left(#1\right]}
\newcommand{\IntRO}[1]{\left[#1\right)}
\newcommand{\Choose}[2]{\left( #1 \atop #2\right)}

\newcommand{\I}{\ensuremath{\mathcal{I}}}
\newcommand{\D}{\ensuremath{\mathbb{D}}}
\DeclareMathOperator*{\argmin}{arg\,min}
\DeclareMathOperator*{\argmax}{arg\,max}
\newcommand{\bigO}{\mathcal{O}}
\newcommand{\omegatilde}{{\tilde{\Omega}}}
\newcommand{\Nin}{\ensuremath{N_{\mathrm{in}}}}
\newcommand{\Nout}{\ensuremath{N_{\mathrm{out}}}}
\newcommand{\din}{\ensuremath{d_{\mathrm{in}}}}
\newcommand{\dout}{\ensuremath{d_{\mathrm{out}}}}

\newcommand{\NP}{$\mathbf{NP}$}
\newcommand{\NPhard}{$\mathbf{NP}$-hard}
\newcommand{\NPcomplete}{$\mathbf{NP}$-complete}
\newcommand{\SPcomplete}{$\mathbf{\#P}$-complete}
\newcommand{\SPhard}{$\mathbf{\#P}$-hard}

\renewcommand{\algorithmicrequire}{\textbf{Input:}}
\renewcommand{\algorithmicensure}{\textbf{Output:}}
\renewcommand{\vec}[1]{\mathbf{#1}}

%% tight spacing in item lists
\newcommand{\squishlist}{
 \begin{list}{$\bullet$}
  {  \setlength{\itemsep}{0pt}
     \setlength{\parsep}{3pt}
     \setlength{\topsep}{3pt}
     \setlength{\partopsep}{0pt}
     \setlength{\leftmargin}{2em}
     \setlength{\labelwidth}{1.5em}
     \setlength{\labelsep}{0.5em}
} }
\newcommand{\squishlisttight}{
 \begin{list}{$\bullet$}
  { \setlength{\itemsep}{0pt}
    \setlength{\parsep}{0pt}
    \setlength{\topsep}{0pt}
    \setlength{\partopsep}{0pt}
    \setlength{\leftmargin}{2em}
    \setlength{\labelwidth}{1.5em}
    \setlength{\labelsep}{0.5em}
} }

\newcommand{\squishdesc}{
 \begin{list}{}
  {  \setlength{\itemsep}{0pt}
     \setlength{\parsep}{3pt}
     \setlength{\topsep}{3pt}
     \setlength{\partopsep}{0pt}
     \setlength{\leftmargin}{1em}
     \setlength{\labelwidth}{1.5em}
     \setlength{\labelsep}{0.5em}
} }

\newcommand{\squishend}{
  \end{list}
}
