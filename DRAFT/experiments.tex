\begin{itemize}
  \item \textbf{RQ1}: Which is the ``best'' method to represent nodes in our model?
  \item \textbf{RQ2}: Is this model tunable for different assumptions?
  \item \textbf{RQ3}: Is this model realistic?
\end{itemize}

\subsection{Interest generation comparison}

How to define the \emph{best} method to represent nodes in our model?

\begin{itemize}
  \item Scalability
  \item Tunable homophily
\end{itemize}

% \begin{figure}[tbp]
% \begin{center}
% \includegraphics[width=\linewidth,draft=true]{plot}
% \caption{Insert plots.}
% \label{fig:plot}
% \end{center}
% \end{figure}

\subsection{Parameters analysis}

Here we fix one or two interests generation methods and we explore the range of properties for the synthetic propagation the model can generate (RQ2).


% \begin{figure}[tbp]
% \begin{center}
% \includegraphics[width=\linewidth,draft=true]{plot}
% \caption{Insert plots.}
% \label{fig:plot}
% \end{center}
% \end{figure}



\subsection{Real data}

Experiments whwere we compare with real data set (RQ3).
